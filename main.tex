% ========================
% PREÁMBULO
% ========================

\documentclass{CEI_slides} % Import document class (CEI_slides.cls)

% ========================
% CUSTOMIZABLE INFORMATION
% ========================
% General information
\newcommand{\speaker}       {Autor Autorez}
\newcommand{\contact}       {autor.autorez@mail.int}
\newcommand{\customdate}    {DD/MM/YYYY}
\newcommand{\titulouno}     {Presentation-style template}
\newcommand{\titulodos}     {in \LaTeX~(Overleaf)}
\newcommand{\titulotres}    {with corporate design}
\newcommand{\titulocuatro}  {from CEI}
\newcommand{\tituloreducido}{Presentation-style template in \LaTeX~with corporate design from CEI}


% ========================
% PRESENTATION
% ========================

\begin{document}

\begin{frame}[plain]
    \titlepage
\end{frame}

\frameNoSubtitle{Itemize format}{
    Itemize environment has been configured to have a square in the first level, a circle in the second level and an arrow in the third one. All bullets are colored using corporative blue.

    \vspace{0.5cm}
    
    \begin{itemize}
        \item Level 1
        \begin{itemize}
            \item Level 2
            \begin{itemize}
                \item Level 3
            \end{itemize}
        \end{itemize}
    \end{itemize}
}


\frameWithSubtitle{Enumerate format}{With subtitle}{
    The enumerate format is quite standard. Corporative blue color has been choose to highlight the levels.

    \vspace{0.5cm}

    \begin{enumerate}
        \item Level 1
        \begin{enumerate}
            \item Level 2
            \begin{enumerate}
                \item Level 3
            \end{enumerate}
        \end{enumerate}
    \end{enumerate}
}

\frameWithSubtitle{Advantages and disadvantages}{Bullet possibilities}{
    In this case, you will need to manually specify each of the bullet points. It is still quite easy :)

    \vspace{0.5cm}
    
    \begin{itemize}
        \item[\advA] First advantage
        \item[\disA] First disadvantage
        \item[\advB] Second advantage
        \item[\disB] Second disadvantage
        \item[\advC] Third advantage
        \item[\disC] Third disadvantage
    \end{itemize}
}

\frameWithSubtitle{Figures and Tables}{\texttt{easyPlacing} function}{
    \easyPlacing{40pt}{80pt}{
        \includegraphics[width=5cm]{example-image-a}
    }
    
    \easyPlacing{250pt}{100pt}{
        \begin{tabular}{l l l}
        \textbf{Name} & \textbf{Age} & \textbf{City} \\ \toprule
        Ana    & 25 & Madrid \\ \hline
        Luis   & 30 & Barcelona \\ \hline
        Marta  & 28 & Valencia \\ \toprule
        \end{tabular}
    }
}


\frameWithSubtitle{Animations}{\texttt{pause} command}{
    \begin{itemize}
        \item To make some text appear, simply use \texttt{pause}. \\
        \pause
        \item This will automatically generate slides to resemble text is appearing sequentially (maintaining the slide number). \\
        \pause
        \item It works with other elements (not if you use \texttt{easyPlacing} command).
    \end{itemize}
}

\framesection{Section title}{Section subtitle}

\framesectiondark{Section title}{Section subtitle}

\frameQA

\framePR

\frameNoSubtitle{Referencing resources}{
    \refleft{9cm}{85pt}{60pt}{6pt}{\sreference}
    \refright{9cm}{85pt}{100pt}{6pt}{\sreference}
    \bookleft{9cm}{85pt}{140pt}{6pt}{\sreference}
    \bookright{9cm}{85pt}{180pt}{6pt}{\sreference}
}

\frameNoSubtitle{Rectangular boxes}{
    \basicbox{7cm}{20pt}{50pt}{ceiDarkBlue}{Abstract}{\stext}
    \basicbox{7cm}{240pt}{50pt}{ceiPurple}{Abstract}{\stext}
    \basicbox{7cm}{20pt}{140pt}{ceiOrange}{Abstract}{\stext}
    \basicbox{7cm}{240pt}{140pt}{ceiGreen}{Abstract}{\stext}

    \polladeagua{5pt}{430pt}{47pt}
    \pitoiberico{3pt}{218.5pt}{80pt}
}

\end{document}
